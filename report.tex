\documentclass[12pt]{article}
\usepackage{amssymb,amsmath,latexsym,amsthm,graphicx}
\begin{document}
\title{CS 698R Final Project Report}
\author{Matthew Webb, Abraham Frandsen}
\date{10 December 2014}

\maketitle

\subsection*{Introduction} (Abe)
We designed and implemented a probabilistic graphical model for image segmentation.
Using UGMs but not learning parameters.


\subsection*{The Model}(Abe)
Before we specify the model, we establish some notation. 
Consider a digital image of dimension $N \times M$.
Assume each pixel value is a $d$-dimensional real-valued vector
(for example, $d$ may be 1 in the case of a Grayscale image, or $d$ may be 3 in the
case of a RGB image), and assume the image contains $K$ segments.
For each $i \in \{1,2\ldots,N\}$ and $j \in \{1,2,\ldots,M\}$,
let $X_{i,j}$ be the numerical value for the $(i,j)$-th pixel, 
and let $Z_{i,j}$ be the (latent) cluster assignment for the pixel. 
Thus, we have $X_{i,j} \in \mathbb{R}^d$ and $Z_{i,j} \in \{1,2,\ldots,K\}$. 
Let $X = (X_{i,j})_{1\leq i \leq N, 1 \leq j \leq M}$ and 
$Z=(Z_{i,j})_{1\leq i \leq N, 1 \leq j \leq M}$. We call $X$ the pixel emissions,
and $Z$ the cluster assignments for the image.

We designed a hybrid directed-undirected graphical joint probability model.

Clique potential is a function of boolean mask.

TODO: show Gibbs sampling updates. 

\subsection*{Data}(Matthew)
talk about images

\subsection*{Design Decisions}(Matthew)

discuss the Gaussian model for pixel emission, as well as the choice for conjugate priors.(abe)

discuss motivations for choosing special clique configurations. 
discuss how this is an attempt to generalize or improve upon Ising model, where
only adjacent pixels are considered.

discuss clever code optimizations.

\subsection*{Results}(Matthew)
Stick in images, discuss parameters used such as segments, factor weights.
Show plots of Gibbs sampler convergence.


\subsection*{Qualitative Evaluation}
strengths, weaknesses, observations.

Cliques are local, so we get fragmented clusters at times. (Matthew)

Intuition about the cluster assignment prior (squeezes out bad shapes). (Matthew)


RGB space different from human qualitative perception, which affects cluster assignments.
Is there an ideal color space corresponding to human perception? (Abe)

Model doesn't account for texture, just assumes each cluster is largely one color (or point 
in RGB space). (Abe)




\subsection*{Conclusion} (Matthew)
Further work.
Joint vs Conditional (CRF).
\end{document}